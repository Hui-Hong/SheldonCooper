\documentclass[a4paper,12pt]{article}
\usepackage{amsmath}  % 数学公式
\usepackage{graphicx} % 插入图片
\usepackage{float}    % 控制图片位置
\usepackage{geometry} % 设置页边距
\usepackage{cite}     % 参考文献支持
\usepackage[colorlinks=true,linkcolor=black,urlcolor=blue]{hyperref}
\usepackage{xcolor}
\geometry{left=2.5cm, right=2.5cm, top=2.5cm, bottom=2.5cm}

% Title部分

\title{Binary Test}
\author{New Team Binary}
\date{\today}

\begin{document}
% 插入标题
\maketitle
\section{Version 1}
Suppose we have two particles 1 and 2. Their masses, positions and velocities are $m_1$ and $m_2$, $\vec{r}_1$ and $\vec{r}_2$, $\vec{v}_1$ and $\vec{v}_1$. The relative position and velocity are $\vec{r}=\vec{r}_1-\vec{r}_2$ $\vec{v}=\vec{v}_1-\vec{v}_2$.

We calculate their reduced mass 
\begin{align}
    \mu=\frac{m_1 m_2}{m_1 +m_2}.
\end{align} 
The position of the center of mass is 
\begin{align}
    \vec{R}_{\rm C}=\frac{m_1 \vec{r}_1+m_2 \vec{r}_2}{m_1+m_2}.
\end{align}

We choose the coordinate where the center of mass is static.
The total energy is 
\begin{align}
     E=\frac{1}{2}\mu v^2-\frac{G m_1 m_2}{r}.\label{eq_Etot}
\end{align}
Here $v=|\vec{v}|$ and $r=|\vec{r}|$.

The angular momentum is 
\begin{align}
    L=\vec{r}\times \mu \vec{v}.
\end{align}

The major axis is
\begin{align}
    a=-\frac{G m_1 m_2}{2E}.
\end{align}

The period is 
\begin{align}
    T^2=\frac{4 \pi^2 a^3}{G(m_1+m_2)}.
\end{align}
    
The eccentricity is 
\begin{align}
    e=\sqrt{1+\frac{2EL^2}{\mu(Gm_1m_2)^2}}.\label{eq_eccen}
\end{align}

The maximum and minimum distance between the two particles are $r_{\rm max}=a(1+e)$ and $r_{\rm min}=a(1-e)$.

The initial conditions we choose are as follows:
$m_1=10^{6}M_{\odot}$,$m_2=10^{6}M_{\odot}$; $\vec{r}_1=(0,0,0)$, $\vec{r}_2=(10^{-5}\,{\rm kpc},0,0)$; $\vec{v}_1=(0,0,0)$, $\vec{v}_2=(0, 10^{-6}\, {\rm kpc/s},0)$.

\section{Version 2}
We generate masses for two particles as \textcolor{red}{$m_1$} and \textcolor{red}{$m_2$}, relative distance \textcolor{red}{$r$}, the eccentricity \textcolor{red}{$e=\frac{c}{a} \in [0.10,0.40)$}.
Here we choose the coordinate at which the center of mass locates at $(0,0,0)$,
\begin{align}
    \vec{R}_{\rm C}&=\frac{m_1 \vec{r}_1+ m_2 \vec{r}_2}{m_1+m_2}\\
    &=0
\end{align}
Thus,
\begin{align}
    \vec{r}_2=-\frac{m_1}{m_2}\vec{r}_1,
\end{align}
and 
\begin{align}
    \vec{r}&=\vec{r}_1-\vec{r}_2\\
    &=\frac{m_1+m_2}{m_2}\vec{r}_1.
\end{align}
Thus,
\textcolor{blue}{
\begin{align}
    \vec{r}_1&=\frac{m_2}{m_1+m_2}\vec{r},\\
    \vec{r}_2&=-\frac{m_1}{m_1+m_2}\vec{r}.
\end{align}
}
If we make C.O.M. (center of mass) to be static,
\begin{align}
    \vec{V}_{\rm C}&=\frac{m_1\vec{v}_1+m_2\vec{v}_2}{m_1+m_2}\\
    &=0.
\end{align}
Similarly,
\begin{align}
    \vec{v}_2=-\frac{m_1}{m_2}\vec{v}_1,
\end{align}
and
\begin{align}
    \vec{v}&=\vec{v}_1-\vec{v}_2\\
    &=\frac{m_1+m_2}{m_2}\vec{v}_1.
\end{align}
Thus,
\textcolor{blue}{
\begin{align}
    \vec{v}_1&=\frac{m_2}{m_1+m_2}\vec{v},\\
    \vec{v}_2&=-\frac{m_1}{m_1+m_2}\vec{v}.
\end{align}
}
According to Eq. \eqref{eq_Etot}) and Eq. \eqref{eq_eccen}, we have
\begin{align}
    E&=\frac{1}{2}\mu v^2-\frac{G m_1 m_2}{r}\\
    &=\frac{1}{2}\frac{m_1 m_2}{m_1+m_2}v^2-\frac{G m_1 m_2}{r},\label{eq_energyfinal}
\end{align}
and 
\begin{align}
    e^2&=1+\frac{2E L^2}{\mu(Gm_1 m_2)^2}.
\end{align}
Suppose $\vec{r}\perp\vec{v}$ initially which means the two particles are separating from each other most or least, we can derive total energy from eccentricity,
\begin{align}
    E&=\frac{\mu\left(G m_1 m_2\right)^2\left(e^2-1\right)}{2L^2}\\
    &=\frac{\mu\left(G m_1 m_2\right)^2\left(e^2-1\right)}{2(r\mu v)^2}\\
    &=\frac{(Gm_1 m_2)^2(e^2-1)}{2\mu (rv)^2}\\
    &=\frac{G^2m_1 m_2(m_1+m_2)(e^2-1)}{2(rv)^2}.\label{eq_energyine}
\end{align}
Combine Eq. \eqref{eq_energyfinal} and Eq. \eqref{eq_energyine}, we have
\begin{align}
    \frac{1}{2}\frac{m_1 m_2}{m_1+m_2}v^2-\frac{G m_1 m_2}{r}&=\frac{G^2m_1 m_2(m_1+m_2)(e^2-1)}{2(rv)^2},
\end{align}
\begin{align}
    \frac{1}{2}\frac{m_1 m_2}{m_1+m_2}v^4-\frac{Gm_1 m_2}{r}v^2-\frac{G^2m_1 m_2(m_1+m_2)(e^2-1)}{2r^2}&=0,\\
    v^4-\frac{2G(m_1+m_2)}{r}v^2-\frac{G^2(m_1+m_2)^2(e^2-1)}{r^2}&=0.
\end{align}
\begin{align}
    v^2&=\frac{\frac{2G(m_1+m_2)}{r}\pm \sqrt{\left[\frac{2G\left(m_1+m_2\right)}{r}\right]^2+4\frac{G^2(m_1+m_2)^2(e^2-1)}{r^2}}}{2}\\
    &=\frac{G(m_1+m_2)}{r}(1\pm e)
\end{align}
Here, `$+$' and `$-$' correspond the smallest and largest initial $r$ respectively.

For given $m_1$, $m_2$, $r$ and $e$,  here we choose
\textcolor{red}{
\begin{align}
    v=\sqrt{\frac{G(m_1+m_2)}{r}(1-e)}.
\end{align}
}
Then total energy \textcolor{red}{$E$} is determined by Eq. \eqref{eq_energyine} and $v$.
For simplicity, we set positions and velocities at $x-y$ plane.
Then 
\textcolor{red}{
\begin{align}
    \vec{r}_1&=(\frac{m_2}{m_1+m_2}r,\, 0,\, 0),\\
    \vec{r}_2&=(-\frac{m_1}{m_1+m_2}r,\, 0,\, 0),\\
    \vec{v}_1&=(0,\, -\frac{m_2}{m_1+m_2}v,\, 0),\\
    \vec{v}_2&=(0,\, \frac{m_1}{m_1+m_2}v,\, 0).
\end{align}
}

\end{document}